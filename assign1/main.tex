\let\negmedspace\undefined
\let\negthickspace\undefined
\documentclass[journal,12pt,onecolumn]{IEEEtran}
\usepackage{cite}
\usepackage{amsmath,amssymb,amsfonts,amsthm}
\usepackage{algorithmic}
\usepackage{graphicx}
\usepackage{textcomp}
\usepackage{xcolor}
\usepackage{txfonts}
\usepackage{listings}
\usepackage{enumitem}
\usepackage{mathtools}
\usepackage{gensymb}
\usepackage[breaklinks=true]{hyperref}
\usepackage{tkz-euclide} % loads  TikZ and tkz-base
\usepackage{listings}



\newtheorem{theorem}{Theorem}[section]
\newtheorem{problem}{Problem}
\newtheorem{proposition}{Proposition}[section]
\newtheorem{lemma}{Lemma}[section]
\newtheorem{corollary}[theorem]{Corollary}
\newtheorem{example}{Example}[section]
\newtheorem{definition}[problem]{Definition}
%\newtheorem{thm}{Theorem}[section] 
%\newtheorem{defn}[thm]{Definition}
%\newtheorem{algorithm}{Algorithm}[section]
%\newtheorem{cor}{Corollary}
\newcommand{\BEQA}{\begin{eqnarray}}
\newcommand{\EEQA}{\end{eqnarray}}
\newcommand{\define}{\stackrel{\triangle}{=}}
\theoremstyle{remark}
\newtheorem{rem}{Remark}
%\bibliographystyle{ieeetr}
\begin{document}
%
\providecommand{\pr}[1]{\ensuremath{\Pr\left(#1\right)}}
\providecommand{\prt}[2]{\ensuremath{p_{#1}^{\left(#2\right)} }}        % own macro for this question
\providecommand{\qfunc}[1]{\ensuremath{Q\left(#1\right)}}
\providecommand{\sbrak}[1]{\ensuremath{{}\left[#1\right]}}
\providecommand{\lsbrak}[1]{\ensuremath{{}\left[#1\right.}}
\providecommand{\rsbrak}[1]{\ensuremath{{}\left.#1\right]}}
\providecommand{\brak}[1]{\ensuremath{\left(#1\right)}}
\providecommand{\lbrak}[1]{\ensuremath{\left(#1\right.}}
\providecommand{\rbrak}[1]{\ensuremath{\left.#1\right)}}
\providecommand{\cbrak}[1]{\ensuremath{\left\{#1\right\}}}
\providecommand{\lcbrak}[1]{\ensuremath{\left\{#1\right.}}
\providecommand{\rcbrak}[1]{\ensuremath{\left.#1\right\}}}
\newcommand{\sgn}{\mathop{\mathrm{sgn}}}
\providecommand{\abs}[1]{\left\vert#1\right\vert}
\providecommand{\res}[1]{\Res\displaylimits_{#1}} 
\providecommand{\norm}[1]{\left\lVert#1\right\rVert}
%\providecommand{\norm}[1]{\lVert#1\rVert}
\providecommand{\mtx}[1]{\mathbf{#1}}
\providecommand{\mean}[1]{E\left[ #1 \right]}
\providecommand{\cond}[2]{#1\middle|#2}
\providecommand{\fourier}{\overset{\mathcal{F}}{ \rightleftharpoons}}
\newenvironment{amatrix}[1]{%
  \left(\begin{array}{@{}*{#1}{c}|c@{}}
}{%
  \end{array}\right)
}
%\providecommand{\hilbert}{\overset{\mathcal{H}}{ \rightleftharpoons}}
%\providecommand{\system}{\overset{\mathcal{H}}{ \longleftrightarrow}}
	%\newcommand{\solution}[2]{\textbf{Solution:}{#1}}
\newcommand{\solution}{\noindent \textbf{Solution: }}
\newcommand{\cosec}{\,\text{cosec}\,}
\providecommand{\dec}[2]{\ensuremath{\overset{#1}{\underset{#2}{\gtrless}}}}
\newcommand{\myvec}[1]{\ensuremath{\begin{pmatrix}#1\end{pmatrix}}}
\newcommand{\mydet}[1]{\ensuremath{\begin{vmatrix}#1\end{vmatrix}}}
\newcommand{\myaugvec}[2]{\ensuremath{\begin{amatrix}{#1}#2\end{amatrix}}}
\providecommand{\rank}{\text{rank}}
\providecommand{\pr}[1]{\ensuremath{\Pr\left(#1\right)}}
\providecommand{\qfunc}[1]{\ensuremath{Q\left(#1\right)}}
	\newcommand*{\permcomb}[4][0mu]{{{}^{#3}\mkern#1#2_{#4}}}
\newcommand*{\perm}[1][-3mu]{\permcomb[#1]{P}}
\newcommand*{\comb}[1][-1mu]{\permcomb[#1]{C}}
\providecommand{\qfunc}[1]{\ensuremath{Q\left(#1\right)}}
\providecommand{\gauss}[2]{\mathcal{N}\ensuremath{\left(#1,#2\right)}}
\providecommand{\diff}[2]{\ensuremath{\frac{d{#1}}{d{#2}}}}
\providecommand{\myceil}[1]{\left \lceil #1 \right \rceil }
\newcommand\figref{Fig.~\ref}
\newcommand\tabref{Table~\ref}
\newcommand{\sinc}{\,\text{sinc}\,}
\newcommand{\rect}{\,\text{rect}\,}
%%
%	%\newcommand{\solution}[2]{\textbf{Solution:}{#1}}
%\newcommand{\solution}{\noindent \textbf{Solution: }}
%\newcommand{\cosec}{\,\text{cosec}\,}
%\numberwithin{equation}{section}
%\numberwithin{equation}{subsection}
%\numberwithin{problem}{section}
%\numberwithin{definition}{section}
%\makeatletter
%\@addtoreset{figure}{problem}
%\makeatother

%\let\StandardTheFigure\thefigure
\let\vec\mathbf

\bibliographystyle{IEEEtran}


\vspace{3cm}



\bigskip

\renewcommand{\thefigure}{\theenumi}
\renewcommand{\thetable}{\theenumi}
%\renewcommand{\theequation}{\theenumi}
Question: Verify that\\
$\frac{BG}{GE}=\frac{CG}{GF}=\frac{AG}{GD}=2$\\
\solution:\\
Let $\vec{D}$,$\vec{E}$,$\vec{F}$ be the midpoints of $BC$,$CA$,$AB$ respectively, then
\begin{align}
\vec{D} = \myvec{\frac{-7}{2}\\\frac{1}{2}}\\
\vec{E} = \myvec{-1\\-3}\\
\vec{F} = \myvec{\frac{-3}{2}\\\frac{5}{2}}
\end{align}
From the previous question 1.2.3, we got
\begin{align}
\vec{G}=\myvec{-2\\0}
\end{align}
Direction vectors as follows: \\
\begin{align}
\vec{B}\vec{G}&=\vec{G}-\vec{B}\\
              &=\myvec{2\\-6}\\
\vec{G}\vec{E}&=\vec{E}-\vec{G}\\
              &=\myvec{1\\-3}\\
\vec{C}\vec{G}&=\vec{G}-\vec{C}\\
              &=\myvec{1\\5}\\
\vec{G}\vec{F}&=\vec{F}-\vec{G}\\
              &=\myvec{\frac{1}{2}\\\frac{5}{2}}\\
\vec{A}\vec{G}&=\vec{G}-\vec{A}\\
              &=\myvec{-3\\1}\\
\vec{G}\vec{D}&=\vec{D}-\vec{G}\\
              &=\myvec{\frac{-3}{2}\\\frac{1}{2}}
\end{align}
Norm of BG and GE:
\begin{align}
\|\mathbf{BG}\|&=\|\mathbf{G-B}\|\\
               &=\sqrt{2^2+(-6)^2}\\
               &=2\sqrt{10}\\
\|\mathbf{GE}\|&=\|\mathbf{G-B}\|\\
               &=\sqrt{1^2+(-3)^2}\\
               &=\sqrt{10}
\end{align}
Norm of CG and GC:\\
\begin{align}
\|\mathbf{CG}\|&=\|\mathbf{G-C}\|\\
               &=\sqrt{1^2+(5)^2}\\
               &=\sqrt{26}\\
\|\mathbf{GC}\|&=\|\mathbf{C-G}\|\\
               &=\sqrt{\left(\frac{1}{2}\right)^2+\left(\frac{5}{2}\right)^2}\\
               &=\frac{1}{2}\sqrt{26}
\end{align}
Norm of AG and GD:\\
\begin{align}
\|\mathbf{AG}\|&=\|\mathbf{G-A}\|\\
               &=\sqrt{(-3)^2+(1)^2}\\
               &=\sqrt{10}\\
\|\mathbf{GD}\|&=\|\mathbf{D-G}\|\\
               &=\sqrt{\left(\frac{-3}{2}\right)^2+\left(\frac{1}{2}\right)^2}\\
               &=\frac{1}{2}\sqrt{10}
\end{align}
The ratios can be calculated as follows:\\
\begin{enumerate} 
\item
\begin{align}
\frac{BG}{GE} =\frac{2\sqrt{10}}{\sqrt{10}}=2
\end{align}
\item 
\begin{align}
\frac{CG}{GF}=\frac{\sqrt{26}}{\frac{1}{2}\sqrt{26}}=2
\end{align}
\item 
\begin{align}
\frac{AG}{GD}=\frac{\sqrt{10}}{\frac{1}{2}\sqrt{10}}=2
\end{align}
\end{enumerate}
Hence,we have verified that \\
$\frac{BG}{GE}=\frac{CG}{GF}=\frac{AG}{GD}=2$
\end{document}
